%==============
%一行目に必ず必要
%文章の形式を定義
%===============
\documentclass[uplatex]{jsbook}
%===============
%パッケージの定義、必要か不明
%===============
%この下4つを加えることで、mathbbが機能した
\usepackage{amsthm}
\usepackage{amsmath}
\usepackage{amssymb}
\usepackage{amsfonts}
%可換図式用パッケージ
\usepackage{amscd}
\usepackage[all]{xy}
\usepackage{tikz-cd}
%リンク用パッケージ
\usepackage[dvipdfmx]{hyperref}
%複数行コメント
%\usepackage{comment}

\usepackage{my-default}
%タイトルデータ
\title{球面調和関数}
\author{ari}
\date{\today}


%===============
%定理環境の設定
%セクション毎
%===============


\begin{document}

% タイトルを出力
\maketitle
% 目次の表示
\tableofcontents

\section{最近の目標}
\begin{itemize}
    \item 数学の応用向けの知識を増加
    \begin{itemize}
        \item 球面調和関数とその既約表現を使って$SE(3)$-equivariantな機械学習モデルを作る
        \item 最適輸送を使ったグラフ等の類似性を調べた研究
    \end{itemize}
    \item 知的好奇心のネタ探し.本当は数論的なアルゴリズムとか面白そうだなと思うけど、それもなあという感じ.
    \item 無気力との戦い.虚しい仕事してると全体的にやる気がなくなる。毎日、生産的に数学をする仕組みが必要.発表がそのモチベーションを一番発揮させやすいかなと思っている.
\end{itemize}

\part{球面調和関数}

ここでの目標は理論的には$SO(3)$の既約表現が球面調和関数を使って決定づけられることである.
可能であればリー群、リー環との関係を明示しながら示したい。
また、それらを用いたいくつかの応用例を調べたい。

最初に調和多項式とLaplacianの定義をもとに調べる.


\chapter{Laplacianと調和多項式}
\section{Laplacianの特徴づけと回転不変な多項式}

\begin{dfn}
$C^{\infty}(\mathbb{K}^n)$に対し,微分作用素$\Delta:= \sum \frac{\partial^2}{\partial x_j^2}$を\textbf{Laplacian}という
また,$\mathrm{Ker} \Delta$の元を調和関数という.特に調和関数であって、多項式であるものを\textbf{調和多項式}という
\end{dfn}

$f \in C^{\infty}(\mathbb{R}^n)$に対して以下の線形作用素を考える
\begin{align*}
\tau_{a}f(x) & := f(x-a) & a \in \mathbb{R}^n \\
T(g)f(x)    & := f(g^{-1}x)  & g \in O(n, \mathbb{R})
\end{align*}

これらは$\Delta:= \sum \frac{\partial^2}{\partial x_j^2}$と可換である.
なぜなら,$\tau_a \Delta f(x) = \Delta \tau_a f(x)$は自明.
$T$については$g$が直交行列であることを使う.
$g$が直交行列のとき
$\frac{\partial }{\partial x_i} f(gx) = \sum_j g_{ij} \frac{\partial f}{\partial x_j} (gx)$となり,
$\frac{\partial^2 }{\partial x_i^2} f(gx) = \sum_k \sum_j  g_{ik} g_{ij} \frac{\partial^2 f}{\partial x_j \partial x_k} (gx)$となる.
よって$\Delta$を作用させたときの$\frac{\partial^2 f}{\partial x_j \partial x_k} (gx)$の係数は$\sum_i g_{ik}g_{ij}$となる.
これは直交行列の性質から$j=k$の時1,$j\neq k$の時0となる.
よって
\begin{equation*}
   \Delta T(g) f(x) = \sum_i \frac{\partial^2}{\partial x_i^2}f(g^{-1}x) = T(g) \Delta f(x) 
\end{equation*}
になる.
$\mathcal{P}_k$で$k$次斉次多項式全体を表すとする.
\begin{prop}
$p \in \mathcal{P}_k^{SO(n, \mathbb{R})}$とする.このとき次が成り立つ.
\begin{align*}
    p(x) =   \begin{cases}
    b ||x||^k &  k \mbox{が偶数のとき} \\
    0 & otherwise
  \end{cases}
\end{align*}
\end{prop}
\begin{proof}
$f(x) := \frac{p(x)}{||x||^k}(x \in \mathbb{R}^k \setminus 0$)とする.
すぐわかることとして,$f(gx) = f(x)$, $f(rx) = f(x)(r \in \mathbb{R})$が言える.
$(x \in \mathbb{R}^k \setminus 0$)に対し,ある$g \in SO(n,\mathbb{R})$が存在し,$g e_n = \frac{x}{||x|| }$ となる.
よって
\begin{equation*}
    f(x) = f(||x|| ge_n)  = f(e_n) = p(e_n) 
\end{equation*}
となり,定数関数である.
よって,$p(x) =  p(e_n) || x||^k$となる.
しかし$k$が奇数のときこれはp$(e_n)=0$を除き多項式でないためこの結果を得る.
\end{proof}


\begin{thm}
 $c_{\alpha} \in C^{\infty}(\mathbb{R}^n)$
 \begin{equation*}
  D:= \sum_{\alpha}  c_{\alpha}(x)(\partial_x)^{\alpha}
 \end{equation*}
 (和は有限和)
この時$D$が
\begin{align*}
\tau_{a} D & = D\tau_{a}  & (a \in \mathbb{R}^n  \\
T(g)D &  = DT(g)  & (\forall g \in SO(n, \mathbb{R})
\end{align*}
をみたすならば,$D$は$\Delta$の多項式である.
\end{thm}
\begin{proof}
1つ目の条件から任意の$f \in C^{\infty}(\mathbb{R})$に対し,
\begin{align*}
    \sum c_{\alpha}(x -a) \partial_x^{\alpha}f(x -a ) = \tau_a D f(x) = D \tau_a f(x) = \sum c_{\alpha}(x) \partial_x^{\alpha}f(x -a )
\end{align*}
となる.
実際$f= x^{\alpha}$とし,$x=a$を代入すると上の式から
\begin{equation*}
    c_{\alpha}(x -a) = c_{\alpha}(x)
\end{equation*}
となる.よって,$c_{\alpha}$は定数関数であることがわかる.
よって$D = q(\partial_x)$となる多項式$q$が取れる.
次に,$f_y(x) = e^{y \cdot x}$を取る.
この置き方から,$q(\partial_x)f_y = q(y) f_y$になる.
また,
\begin{equation*}
T(g) f_y(x) = e^{y^Tg^Tx} =  e^{gy \cdot x} = f_{gy}(x)
\end{equation*} 
に注意すると,
\begin{align*}
    T(g)D f_y(x) & = T(g) q(y) f_y = q(y) f_{gy}  \\
    = DT(g) f_y(x) & =q(\partial_x) f_{gy} = q(gy) f_{gy}
\end{align*}
となる.よって$q(y) = q(gy)$となるので,
$q \in \oplus \mathcal{P}_k^{SO(n, \mathbb{R})}$となる.
よって上の補題から,
$q(x) =  \sum_m b_m ||x||^{2m}$とかける.
よって$p(t) = \sum_m b_m t^m$とすると
$p(\Delta) = q(\partial_x) = D$となる.
\end{proof}

\begin{thebibliography}{99}
\bibitem{K}  関数解析(黒田)
\end{thebibliography}
\end{document}
